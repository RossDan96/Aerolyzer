\documentclass[onecolumn, draftclsnofoot,10pt, compsoc]{IEEEtran}
\usepackage{graphicx}
\usepackage{url}
\usepackage{setspace}

\usepackage{geometry}
\geometry{textheight=9.5in, textwidth=7in}

% 1. Fill in these details
\def \CapstoneTeamName{		Aerolyzer}
\def \CapstoneTeamNumber{		19}
\def \GroupMemberOne{			Daniel Ross}
\def \GroupMemberTwo{			Kin-Ho Lam}
\def \GroupMemberThree{			Logan Wingard}
\def \CapstoneProjectName{		Aerolyzer}
\def \CapstoneSponsorCompany{	NASA JPL}
\def \CapstoneSponsorPerson{		Kim Whitehall}


% 2. Uncomment the appropriate line below so that the document type works
\def \DocType{		%Problem Statement
				%Requirements Document
				Technology Review
				%Design Document
				%Progress Report
				}
			
\newcommand{\NameSigPair}[1]{\par
\makebox[2.75in][r]{#1} \hfil 	\makebox[3.25in]{\makebox[2.25in]{\hrulefill} \hfill		\makebox[.75in]{\hrulefill}}
\par\vspace{-12pt} \textit{\tiny\noindent
\makebox[2.75in]{} \hfil		\makebox[3.25in]{\makebox[2.25in][r]{Signature} \hfill	\makebox[.75in][r]{Date}}}}
% 3. If the document is not to be signed, uncomment the RENEWcommand below
\renewcommand{\NameSigPair}[1]{#1}

%%%%%%%%%%%%%%%%%%%%%%%%%%%%%%%%%%%%%%%
\begin{document}
\begin{titlepage}
    \pagenumbering{gobble}
    \begin{singlespace}
    	\includegraphics[height=4cm]{coe_v_spot1}
        \hfill 
        % 4. If you have a logo, use this includegraphics command to put it on the coversheet.
        %\includegraphics[height=4cm]{CompanyLogo}   
        \par\vspace{.2in}
        \centering
        \scshape{
            \huge CS Capstone \DocType \par
            {\large\today}\par
            \vspace{.5in}
            \textbf{\Huge\CapstoneProjectName}\par
            \vfill
            {\large Prepared for}\par
            \Huge \CapstoneSponsorCompany\par
            \vspace{5pt}
            {\Large\NameSigPair{\CapstoneSponsorPerson}\par}
            {\large Prepared by }\par
            \GroupMemberOne\par
            % 5. comment out the line below this one if you do not wish to name your team
            \CapstoneTeamName\par 
            \vspace{5pt}
            {\large
                \NameSigPair{\GroupMemberOne}\par
                \NameSigPair{\GroupMemberTwo}\par
                \NameSigPair{\GroupMemberThree}\par
            }
            \vspace{20pt}
        }
        \begin{abstract}
        % 6. Fill in your abstract    
        	Virtualbox is a straight forward solution for development environment.
			Django is ideal for python library integration.
			CanvasJS is the most appealing visualization tool.
        \end{abstract}     
    \end{singlespace}
\end{titlepage}
\newpage
\pagenumbering{arabic}
\tableofcontents
% 7. uncomment this (if applicable). Consider adding a page break.
%\listoffigures
%\listoftables
\clearpage

\begin{singlespace}
\section{Database Management System}
\subsection{Overview}

\subsection{Criteria}
\begin{enumerate}
\item Meets ACID Properties
\item Speed
\item Size
\end{enumerate}
\subsection{Options}
\subsubsection{SQLite3}

\subsubsection{MySQL}

\subsubsection{PostgreSQL}

\subsection{Database Management System Discussion}

\subsection{Conclusion}


\section{Web Framework}
\subsection{Overview}
Web frameworks are what will enable the Aerolyzer library to interact with the users, so the framework's performance will directly affect the end product of the Aerolyzer app.
\subsection{Criteria}
\begin{enumerate}
\item Speed
\item Security
\item Web Development features
\end{enumerate}
\subsection{Options}
\subsubsection{Express.Js}
Express is a streamlined version of Node.js.
Express itself doesn't have database support, but is capable of running third-party modules to interface with the database.
Express is useful for API development as it includes the neccesary http functionality to provide users functions.
\subsubsection{Ruby on Rails}
 Ruby on Rails, 'Rails' for short, is a Ruby based web framework.
Rails claims to value web application conventions and will default to many of those conventions in the case that the developer didn't make custom configurations.
This hopes to make Rails code shorter and less buggy than other frameworks that require minutiae from the developer.
Rails runs off of a SQLite3 database and is best suited for interfacing with that type of database.
\subsubsection{Django}
Django is a python based framework that claims to make database-driven Web development fast and easy.
The first step to using Django is writing the model which maps out your database layout.
Django also supports clean URL designs and in framework templating.
\subsection{Web Framework Discussion}
The Aerolyzer project already has a Django web application developed, and this is ideal because integrating the Aerolyzer Python Library into Rails or Express would be more difficult. 
\subsection{Conclusion}
blah


\section{Visualization \& Display}
\subsection{Overview}
Presenting the output from the Aerolyzer library is the key step in making the user understand the project.
Having unclear or unappealing visuals to show the results could potentially be worse than simple outputs.
Most of the presentation can be acheived with HTML alone, but the main display will need a script to insert charts.
\subsection{Criteria}
\begin{enumerate}
\item Clarity
\item Reliability
\end{enumerate}
\subsection{Options}
\subsubsection{Google Charts}
Google Charts is a JavaScript that is inserted into the web page and is then fed the data to be charted.
There are 28 different types of charts and they all are rendered as SVG or VML.
Google Charts lets the chart make a query to the database directly and displays customizable tooltips when moused over.
\subsubsection{CanvasJS}
CanvasJS, in addition to having 30 types of charts, claims to be an ultra fast HTML5 charting library.
Rather than render SVG charts, CanvasJS inserts HTML code when rendered.
With a wide library of customization, CanvasJS claims to make the developer's charts work on Chrome, Firefox, Safari, and Internet Explorer.
\subsubsection{ChartJS}
ChartJS renders charts into HTML5, like CanvasJS, but is open source.
ChartJS only has 8 chart types, but offers custimization that lets developers mix chart types.
\subsection{Visualization \& Display Discussion}
CanvasJS is the most robust JavaScript of the 3, which may take some additional code to get the desired results, but the end product will most likely be closer to the ideal visualization. 
\subsection{Conclusion}
blah


\begin{center}
 \begin{tabular}{|l|c|c|} 
 \hline
 \multicolumn{3}{|c|}{Database Management System} \\
 \hline
 Choice & Pros & Cons\\ [0.5ex] 
 \hline\hline
 SQLite3 & 6 & 87837 \\ 
 \hline
 MySQL & 545 & 778\\
 \hline
 PostgreSQL & 88 & 788\\ [1ex] 
 \hline
\end{tabular}
 \begin{tabular}{|l|c|c|} 
 \hline
 \multicolumn{3}{|c|}{Web Framework} \\
 \hline
 Choice & Pros & Cons\\ [0.5ex] 
 \hline\hline
 Express.js & 6 & 87837 \\ 
 \hline
 Ruby on Rails & 545 & 778\\
 \hline
 Django & 88 & 788\\ [1ex] 
 \hline
\end{tabular}
 \begin{tabular}{|l|c|c|} 
 \hline
 \multicolumn{3}{|c|}{Visualization \& Display} \\
 \hline
 Choice & Pros & Cons\\ [0.5ex] 
 \hline\hline
 Google Charts & 6 & 87837 \\ 
 \hline
 ChartJS & 545 & 778\\
 \hline
 CanvasJS & 88 & 788\\ [1ex] 
 \hline
\end{tabular}
\end{center}

\bibliographystyle{IEEEtran}
\bibliography{ref}


\end{singlespace}
\end{document}
